\documentclass[a4wide, 10pt]{article}
\usepackage{a4, fullpage}
\usepackage[section]{placeins}
\setlength{\parskip}{0.3cm}
\setlength{\parindent}{0.5cm}

\begin{document}

\title{Group 27 Emulator Summary}

\author{William Springsteen \and Rory Fayed \and Stefan Klas \and Kiran Patel}

\maketitle

At the start of the project, we decided that we would wait until we learned a bit more
 of the C language, and had a bit more practice in it, to start writing any code. This
  meant that we were going to wait until after the weekend to meet in labs and really
   crack on with writing the emulate program. Over the weekend, however, our group
    communicated and it was decided that Stefan and Billy would have a go at writing
     the outline of the emulate.c file. In the meantime, Kiran and Rory would carefully
      go through the specification, learn some more C, and get some more practice in C.
       They would also look at the code that Stefan and Billy would regularly push to
        git, as more code was written, and give feedback as we went along. All
         communication over the weekend occurred via online messages.

After the weekend, we decided to meet in the University labs at 10am, and had a talk
 about everything we had found out/worked on over the weekend. We had to make sure that
  everybody was on the same level of understanding on the code and how it worked, and
   we had to all agree on the way we were going to go about writing the program. For
    example, with respect to reading the instructions from the binary file, the
     instructions are little endian in the file, and we had to decide whether to store
      this in the memory in little endian form or big endian form. Another thing we had
       to agree on was what form we would put each instruction in so that we could
        manipulate it in various methods – we could either use a normal number in
         binary form, use an array of ‘1’ or ‘0’ chars, or use an array of 1 or 0 ints.
          At the time, we didn’t realise that any int, or \texttt{uint32\_t} type could be
           converted to binary using ‘0b’, so we thought we had to use the command
            ‘\texttt{uint32\_t} x = 1011’ to make x have the binary value 1011, or 11 in base 10,
             and we obviously knew this would break when there were more than 10 bits,
              or some value around that number, so there was no way we could store a
               32-bit value like this. However, we now know that ‘\texttt{uint32\_t} x = 11’ is
                the same as ‘\texttt{uint32\_t} x = 0b1011’. After making sure we were all doing
                 things the same way, we shared/brainstormed ideas for the rest of the
                  program, and decided who wanted to do what parts of the program,
                   after deciding to use mostly Stefan’s current layout (with enums and
                    structs too), and mostly Billy’s current implementations. We
                     decided that for the next few days, Billy would work on decode,
                      other general small bits, and make sure everybody was getting on
                       ok with everything, Stefan would work on decode and execute,
                        Kiran would work on file loading and main, and Rory would work
                         on execute and other bits that people needed help on.

After we all finished our parts, all that was really left to do was debugging and
 testing. We decided it would be best to also split up our 1000 line file into lots of
  smaller .c files and header files, which would both make it easier to debug and make
   the code easier to read, more presentable, and more professional. Kiran and Rory
    were assigned to do the file splitting, while Stefan and Billy were assigned to do
     the testing and debugging. The initial debugging involved mainly syntactical
      errors, such as missing semicolons, as well as spelling mistakes on some
       variables/method names, so that we could actually compile the emulator. Once
        Kiran and Rory had finished the file splitting, they started solely working on
         testing and debugging too, as there was a lot of code to debug, and bugs can
          be hard to find and fix.

Overall, the group is working really well together. There has been lots of
 communication between all members, both when not together (via online group messages),
  and when meeting in labs. We were all willing to meet in labs every day after the
   weekend from the morning until the evening, and sometimes some members would stay
    even later than that. We always made sure that everybody was kept up to speed on
     what we had all coded, and any new ideas any of us had. The only thing that might
      need to change is when we decide to really start the task. This is because, for
       emulate, we decided not to do too much until after the weekend, which we found
        meant that we were a little bit pushed for time to finish by Friday, as emulate
         took much longer than we had hoped. We would then have had more time to think
          about how to do part 2 (assemble) while making emulate, which might have
           changed the way we made the emulator, because we might have done something
            in a different way so that we could reuse some code/ideas for assemble.
            
For our emulator, we have some global constants made using \texttt{\#define}, for things like the
 number of registers and the size of the memory, which stop magic numbers being used. We have
  lots of enumerated types, for the different conditions, the positions of the important CPSR
   flag bits, the different opcodes, the different shift codes, and the different instruction
    types (Data processing, branch, halt, cleared, etc.). These enumerated types help make it
     easier to tell what the different numbers mean when we are processing them. For example if
      we process the opcode as being 0xD for a particular data processing command, we know it is
       a MOV command, thanks to the enumerated type, and we can check this by saying /texttt{if
        (0xD == MOV)}. We have a big struct for the virtual ARM machine, which has an array of of
         8-bit ints for the memory, as the memory is byte addressable, and an array of 32-bit
          ints for the values in the registers. We have a struct for the final fetched
           instruction and the final decoded instruction, each of which will only have one
            instance created, which will be the two parts of the 'pipeline'. The fetched struct
             has a \texttt{uint32\_t} for the actual fetched instruction itself, and a bool
              called isFetchedCleared to be able to tell if the fetched stage in the pipeline is
               cleared. The decoded struct has an instruction type enum, so the program can tell
                what type of instruction it is, as well as if the decoded part of the pipeline is
                 cleared, and a union of the decoded instruction's information. This union of the
                  decoded instruction's information can be one of four structs (one for each type
                   of instruction that isn't halt or cleared), each of which holds the
                    information for each different instruction type, such as the condition enum,
                     the opcode enum, the offset, the different register numbers, etc. To tell
                      which one of these structs is set in the union at any given time, you can
                       check the instruction type in the struct which contains the decoded union
                        and the decoded instruction type. 
                        
                        The most important functions we have
                         are fetch, decode and execute, as well as main, obviously. Fetch takes
                          into account the litte endianness of the memory, converts the next
                           instruction into big endian format, and stores it in the fetched part
                            of the 'pipeline'. Decode changes the decoded part of the 'pipeline',
                             and will decode any instruction it is given, although the program
                              has been made so that decode doesn't handle the case where the
                               decode part of the 'pipeline' is cleared, and instead decode
                                simply won't be called if this is the case. Execute will execute
                                 the function that is in the decoded part of the 'pipeline'. Lots
                                  of helper functions are used throughout these big functions.
                                   The main function will load every instruction into memory in
                                    little endian form, and will then proceed to execute, decode,
                                     fetch and increment the PC register, in that order, in a
                                      loop, until a halt command is reached (all 0 instruction).
                                       Upon termination, the program will print out its registers
                                        and non-zero memory. We decided to split up all these
                                         methods into different files, so that it was easier to
                                          find specific methods, and it looked a lot neater. All
                                           methods were grouped together and put in a file with a
                                            logical name, so that all related methods are
                                             together as much as possible.
                                      
We aren't really sure how much we can reuse for the assembler, but the fact that we split our
 emulator into lots of different methods/helper methods means that we should be able to reuse a
  lot of these helper methods, such as the methods in the binMethods.c file, which are methods
   dealing with the manipulation of binary numbers. 

We think we may have problems getting the next parts finished on time, especially as we think the
 extension will take quite a long time, so we are going to try to start everything early and not
  waste any time. We also found that there are a few things we were unsure on, and found it
   difficult to understand them just by looking on the Internet, as sometimes things can be
    ambiguous. Due to this, we will attempt to ask lab helpers or members of staff or other
     students for help understanding things if we are really stuck, which will hopefully stop us
      doing something wrong at the start of a program, and not realising until the end, which
       would cause the changing of a lot of code.

\end{document}